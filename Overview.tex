\documentclass{article}
\usepackage[utf8]{inputenc}
\usepackage{amsmath}

\title{Electric Field Visualization}
\author{}
\date{}

\begin{document}

\maketitle

This document provides an overview of the Python code that generates a plot illustrating the electric field lines produced by a set of charges distributed on a unit circle. Let's break down the code step by step:

\section{Importing Libraries}

The code begins by importing necessary libraries: NumPy for numerical computations and matplotlib for plotting.

\section{Electric Field Function}

The function $E(q, r_0, x, y)$ computes the electric field vector $(E_x, E_y)$ at a point $(x, y)$ due to a charge $q$ located at position $(r_{0x}, r_{0y})$. It uses Coulomb's law to calculate the electric field. The formula used is:

\[
E_x = \frac{q(x - r_{0x})}{|\vec{r} - \vec{r_0}|^3}
\]
\[
E_y = \frac{q(y - r_{0y})}{|\vec{r} - \vec{r_0}|^3}
\]

\section{Grid of Points}

The code creates a grid of points in the $x$ and $y$ directions using \texttt{np.linspace()} and \texttt{np.meshgrid()} functions.

\section{Charges}

It generates $n_q$ charges distributed equally spaced on a unit circle. Each charge alternates in sign between positive and negative. The charges are stored as tuples containing the charge value and its position on the unit circle.

\section{Electric Field Calculation}

It calculates the electri
